% Options for packages loaded elsewhere
\PassOptionsToPackage{unicode}{hyperref}
\PassOptionsToPackage{hyphens}{url}
%
\documentclass[
]{article}
\usepackage{amsmath,amssymb}
\usepackage{lmodern}
\usepackage{iftex}
\ifPDFTeX
  \usepackage[T1]{fontenc}
  \usepackage[utf8]{inputenc}
  \usepackage{textcomp} % provide euro and other symbols
\else % if luatex or xetex
  \usepackage{unicode-math}
  \defaultfontfeatures{Scale=MatchLowercase}
  \defaultfontfeatures[\rmfamily]{Ligatures=TeX,Scale=1}
\fi
% Use upquote if available, for straight quotes in verbatim environments
\IfFileExists{upquote.sty}{\usepackage{upquote}}{}
\IfFileExists{microtype.sty}{% use microtype if available
  \usepackage[]{microtype}
  \UseMicrotypeSet[protrusion]{basicmath} % disable protrusion for tt fonts
}{}
\makeatletter
\@ifundefined{KOMAClassName}{% if non-KOMA class
  \IfFileExists{parskip.sty}{%
    \usepackage{parskip}
  }{% else
    \setlength{\parindent}{0pt}
    \setlength{\parskip}{6pt plus 2pt minus 1pt}}
}{% if KOMA class
  \KOMAoptions{parskip=half}}
\makeatother
\usepackage{xcolor}
\usepackage[margin=1in]{geometry}
\usepackage{color}
\usepackage{fancyvrb}
\newcommand{\VerbBar}{|}
\newcommand{\VERB}{\Verb[commandchars=\\\{\}]}
\DefineVerbatimEnvironment{Highlighting}{Verbatim}{commandchars=\\\{\}}
% Add ',fontsize=\small' for more characters per line
\usepackage{framed}
\definecolor{shadecolor}{RGB}{248,248,248}
\newenvironment{Shaded}{\begin{snugshade}}{\end{snugshade}}
\newcommand{\AlertTok}[1]{\textcolor[rgb]{0.94,0.16,0.16}{#1}}
\newcommand{\AnnotationTok}[1]{\textcolor[rgb]{0.56,0.35,0.01}{\textbf{\textit{#1}}}}
\newcommand{\AttributeTok}[1]{\textcolor[rgb]{0.77,0.63,0.00}{#1}}
\newcommand{\BaseNTok}[1]{\textcolor[rgb]{0.00,0.00,0.81}{#1}}
\newcommand{\BuiltInTok}[1]{#1}
\newcommand{\CharTok}[1]{\textcolor[rgb]{0.31,0.60,0.02}{#1}}
\newcommand{\CommentTok}[1]{\textcolor[rgb]{0.56,0.35,0.01}{\textit{#1}}}
\newcommand{\CommentVarTok}[1]{\textcolor[rgb]{0.56,0.35,0.01}{\textbf{\textit{#1}}}}
\newcommand{\ConstantTok}[1]{\textcolor[rgb]{0.00,0.00,0.00}{#1}}
\newcommand{\ControlFlowTok}[1]{\textcolor[rgb]{0.13,0.29,0.53}{\textbf{#1}}}
\newcommand{\DataTypeTok}[1]{\textcolor[rgb]{0.13,0.29,0.53}{#1}}
\newcommand{\DecValTok}[1]{\textcolor[rgb]{0.00,0.00,0.81}{#1}}
\newcommand{\DocumentationTok}[1]{\textcolor[rgb]{0.56,0.35,0.01}{\textbf{\textit{#1}}}}
\newcommand{\ErrorTok}[1]{\textcolor[rgb]{0.64,0.00,0.00}{\textbf{#1}}}
\newcommand{\ExtensionTok}[1]{#1}
\newcommand{\FloatTok}[1]{\textcolor[rgb]{0.00,0.00,0.81}{#1}}
\newcommand{\FunctionTok}[1]{\textcolor[rgb]{0.00,0.00,0.00}{#1}}
\newcommand{\ImportTok}[1]{#1}
\newcommand{\InformationTok}[1]{\textcolor[rgb]{0.56,0.35,0.01}{\textbf{\textit{#1}}}}
\newcommand{\KeywordTok}[1]{\textcolor[rgb]{0.13,0.29,0.53}{\textbf{#1}}}
\newcommand{\NormalTok}[1]{#1}
\newcommand{\OperatorTok}[1]{\textcolor[rgb]{0.81,0.36,0.00}{\textbf{#1}}}
\newcommand{\OtherTok}[1]{\textcolor[rgb]{0.56,0.35,0.01}{#1}}
\newcommand{\PreprocessorTok}[1]{\textcolor[rgb]{0.56,0.35,0.01}{\textit{#1}}}
\newcommand{\RegionMarkerTok}[1]{#1}
\newcommand{\SpecialCharTok}[1]{\textcolor[rgb]{0.00,0.00,0.00}{#1}}
\newcommand{\SpecialStringTok}[1]{\textcolor[rgb]{0.31,0.60,0.02}{#1}}
\newcommand{\StringTok}[1]{\textcolor[rgb]{0.31,0.60,0.02}{#1}}
\newcommand{\VariableTok}[1]{\textcolor[rgb]{0.00,0.00,0.00}{#1}}
\newcommand{\VerbatimStringTok}[1]{\textcolor[rgb]{0.31,0.60,0.02}{#1}}
\newcommand{\WarningTok}[1]{\textcolor[rgb]{0.56,0.35,0.01}{\textbf{\textit{#1}}}}
\usepackage{graphicx}
\makeatletter
\def\maxwidth{\ifdim\Gin@nat@width>\linewidth\linewidth\else\Gin@nat@width\fi}
\def\maxheight{\ifdim\Gin@nat@height>\textheight\textheight\else\Gin@nat@height\fi}
\makeatother
% Scale images if necessary, so that they will not overflow the page
% margins by default, and it is still possible to overwrite the defaults
% using explicit options in \includegraphics[width, height, ...]{}
\setkeys{Gin}{width=\maxwidth,height=\maxheight,keepaspectratio}
% Set default figure placement to htbp
\makeatletter
\def\fps@figure{htbp}
\makeatother
\setlength{\emergencystretch}{3em} % prevent overfull lines
\providecommand{\tightlist}{%
  \setlength{\itemsep}{0pt}\setlength{\parskip}{0pt}}
\setcounter{secnumdepth}{-\maxdimen} % remove section numbering
\ifLuaTeX
  \usepackage{selnolig}  % disable illegal ligatures
\fi
\IfFileExists{bookmark.sty}{\usepackage{bookmark}}{\usepackage{hyperref}}
\IfFileExists{xurl.sty}{\usepackage{xurl}}{} % add URL line breaks if available
\urlstyle{same} % disable monospaced font for URLs
\hypersetup{
  pdftitle={Analisis de regresion},
  pdfauthor={Ronald Bailey},
  hidelinks,
  pdfcreator={LaTeX via pandoc}}

\title{Analisis de regresion}
\author{Ronald Bailey}
\date{2023-05-25}

\begin{document}
\maketitle

\hypertarget{laboratorio-3-modelos-de-regresiuxf3n-en-r}{%
\section{\texorpdfstring{\textbf{Laboratorio \#3: Modelos de Regresión
en
R}}{Laboratorio \#3: Modelos de Regresión en R}}\label{laboratorio-3-modelos-de-regresiuxf3n-en-r}}

\textbf{Ejercicio \#1:} utilizando R realice una función que dado un
dataframe cualquiera de dos columnas, donde la primera (índice 1) sea el
valor de la variable independiente (X) y la segunda sea el valor de una
variable dependiente (Y), devuelva una lista con los siguientes
elementos:

\begin{enumerate}
\def\labelenumi{\arabic{enumi}.}
\tightlist
\item
  Un arreglo con los valores de los estimadores para B0 y B1''.
\item
  El valor del coeficiente de determinación r\^{}2 del modelo.
\item
  El coeficiente de correlación r (raíz cuadrada de r\^{}2).
\item
  Un arreglo con los valores de los residuos.
\item
  Una gráfica con la nube de puntos y la recta de regresión del modelo.
\end{enumerate}

\textbf{Nota:} Para este ejercicio NO está permitido utilizar la función
lm()para calcular ninguno de los elementos solicitados (incisos 1 al 4),
sin embargo puede utilizar ggplot para realizar la gráfica del inciso 5

Recuerde de su curso de Econometria que: \[
\hat{\beta}_1 = \frac{\sum _{i=1}^nx_iy_i -n \sum _{i=1}^nx_iy_i}{(\sum _{i=1}^n x_i )^2 - n\sum _{i=1}^n x_i^2}
\] \[
\hat{\beta}_0 = \frac{\sum _{i=1}^ny_i -\beta_1 \sum _{i=1}^nx_i}{n}
\] \[
r^2= \frac{\sum _{i=1}(\hat y_i - \overline{y_i} )^2 }{ \sum _{i=1}( y_i - \overline{y_i} )^2}
\]

\[
r= \sqrt{r}
\]

\textbf{Recuerde que � representa la cantidad de filas en el dataset}

\hypertarget{leer-el-dataset}{%
\section{Leer el dataset}\label{leer-el-dataset}}

\begin{Shaded}
\begin{Highlighting}[]
\NormalTok{dataset }\OtherTok{=} \FunctionTok{read.csv}\NormalTok{(}\StringTok{"admisions.csv"}\NormalTok{)}

\FunctionTok{head}\NormalTok{(dataset)}
\end{Highlighting}
\end{Shaded}

\begin{verbatim}
##   Serial.No. GRE.Score TOEFL.Score University.Rating SOP LOR CGPA Research
## 1          1       337         118                 4 4.5 4.5 9.65        1
## 2          2       324         107                 4 4.0 4.5 8.87        1
## 3          3       316         104                 3 3.0 3.5 8.00        1
## 4          4       322         110                 3 3.5 2.5 8.67        1
## 5          5       314         103                 2 2.0 3.0 8.21        0
## 6          6       330         115                 5 4.5 3.0 9.34        1
##   Chance.of.Admit
## 1            0.92
## 2            0.76
## 3            0.72
## 4            0.80
## 5            0.65
## 6            0.90
\end{verbatim}

\end{document}
