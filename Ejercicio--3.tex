% Options for packages loaded elsewhere
\PassOptionsToPackage{unicode}{hyperref}
\PassOptionsToPackage{hyphens}{url}
%
\documentclass[
]{article}
\usepackage{amsmath,amssymb}
\usepackage{lmodern}
\usepackage{iftex}
\ifPDFTeX
  \usepackage[T1]{fontenc}
  \usepackage[utf8]{inputenc}
  \usepackage{textcomp} % provide euro and other symbols
\else % if luatex or xetex
  \usepackage{unicode-math}
  \defaultfontfeatures{Scale=MatchLowercase}
  \defaultfontfeatures[\rmfamily]{Ligatures=TeX,Scale=1}
\fi
% Use upquote if available, for straight quotes in verbatim environments
\IfFileExists{upquote.sty}{\usepackage{upquote}}{}
\IfFileExists{microtype.sty}{% use microtype if available
  \usepackage[]{microtype}
  \UseMicrotypeSet[protrusion]{basicmath} % disable protrusion for tt fonts
}{}
\makeatletter
\@ifundefined{KOMAClassName}{% if non-KOMA class
  \IfFileExists{parskip.sty}{%
    \usepackage{parskip}
  }{% else
    \setlength{\parindent}{0pt}
    \setlength{\parskip}{6pt plus 2pt minus 1pt}}
}{% if KOMA class
  \KOMAoptions{parskip=half}}
\makeatother
\usepackage{xcolor}
\usepackage[margin=1in]{geometry}
\usepackage{graphicx}
\makeatletter
\def\maxwidth{\ifdim\Gin@nat@width>\linewidth\linewidth\else\Gin@nat@width\fi}
\def\maxheight{\ifdim\Gin@nat@height>\textheight\textheight\else\Gin@nat@height\fi}
\makeatother
% Scale images if necessary, so that they will not overflow the page
% margins by default, and it is still possible to overwrite the defaults
% using explicit options in \includegraphics[width, height, ...]{}
\setkeys{Gin}{width=\maxwidth,height=\maxheight,keepaspectratio}
% Set default figure placement to htbp
\makeatletter
\def\fps@figure{htbp}
\makeatother
\setlength{\emergencystretch}{3em} % prevent overfull lines
\providecommand{\tightlist}{%
  \setlength{\itemsep}{0pt}\setlength{\parskip}{0pt}}
\setcounter{secnumdepth}{-\maxdimen} % remove section numbering
\ifLuaTeX
  \usepackage{selnolig}  % disable illegal ligatures
\fi
\IfFileExists{bookmark.sty}{\usepackage{bookmark}}{\usepackage{hyperref}}
\IfFileExists{xurl.sty}{\usepackage{xurl}}{} % add URL line breaks if available
\urlstyle{same} % disable monospaced font for URLs
\hypersetup{
  pdftitle={Ejercicio \# 3},
  pdfauthor={Ronald Bailey},
  hidelinks,
  pdfcreator={LaTeX via pandoc}}

\title{Ejercicio \# 3}
\author{Ronald Bailey}
\date{2023-05-28}

\begin{document}
\maketitle

\hypertarget{ejercicio-3}{%
\section{\texorpdfstring{\textbf{Ejercicio
\#3:}}{Ejercicio \#3:}}\label{ejercicio-3}}

A continuación se le muestran tres imágenes que muestran los resultados
obtenidos de correr la función summary() a dos modelos de regresión
lineal, para este ejercicio se le solicita que realice la interpretación
de las tablas resultantes. Recuerde tomar en cuenta la signficancia de
los parámetros (signfícancia local), la signficancia del modelo
(signficancia global), el valor del �!: y cualquier observación que
considere relevante para determinar si el modelo estructuralmente es
adecuado o no.

\hypertarget{modelo-1}{%
\section{\texorpdfstring{\textbf{\emph{Modelo
\#1}}}{Modelo \#1}}\label{modelo-1}}

La fórmula del modelo es ROLL \textasciitilde{} UNEM, lo que indica que
la variable de respuesta es ``ROLL'' y la variable predictora es
``UNEM''.

Los coeficientes del modelo son los siguientes:

\begin{itemize}
\item
  Intercepto (Intercept): El valor del intercepto es 3957. Esto
  significa que cuando la variable UNEM es cero, se espera que la
  variable de respuesta ROLL sea 3957. El valor p asociado con este
  coeficiente es 0.3313, que es mayor que 0.05, lo que significa que
  este coeficiente no es estadísticamente significativo al nivel del
  5\%. En otras palabras, no hay evidencia suficiente para rechazar la
  hipótesis nula de que este coeficiente sea cero.
\item
  UNEM: El coeficiente para UNEM es 1133.8, lo que significa que por
  cada incremento unitario en UNEM, se espera que ROLL aumente en 1133.8
  unidades, asumiendo que todas las demás variables se mantienen
  constantes. El valor p asociado con este coeficiente es 0.0358, que es
  menor que 0.05, indicando que este coeficiente es estadísticamente
  significativo al nivel del 5\%.
\end{itemize}

El modelo tiene un R-cuadrado de 0.1531, lo que indica que el modelo
explica el 15.31\% de la variabilidad en la variable de respuesta ROLL.
El R-cuadrado ajustado es 0.1218, que toma en cuenta el número de
predictores en el modelo (en este caso, solo uno).

El valor F es 4.883 con un valor p de 0.03579. Esto indica que el modelo
en su conjunto es estadísticamente significativo al nivel del 5\%, es
decir, hay evidencia de que al menos uno de los coeficientes del modelo
es distinto de cero.

En general, aunque el modelo es significativo, el R-cuadrado es bastante
bajo, lo que indica que el modelo no explica una gran cantidad de la
variabilidad en los datos. Además, solo la variable predictora UNEM es
significativa, mientras que el intercepto no es significativo. Esto
podría sugerir que el modelo puede no ser el mejor ajuste para estos
datos y que podrían ser necesarios más predictores o un modelo más
complejo.

\hypertarget{modelo-2}{%
\section{Modelo \#2:}\label{modelo-2}}

Intercepto (Intercept): El intercepto del modelo es -1.338e+03. Sin
embargo, debido a que estamos trabajando con un modelo de regresión
múltiple, la interpretación del intercepto es menos directa. Aquí,
representa el valor esperado de la variable de respuesta ROLL cuando
todas las variables predictoras son cero. Dado que el valor p asociado
es significativo (5.02e-09, menor que 0.05), podemos decir que este
coeficiente es estadísticamente significativo.

UNEM: El coeficiente de UNEM es -9.153e+03, lo que indica que por cada
incremento unitario en UNEM, se espera que ROLL disminuya en 9.153e+03
unidades, manteniendo constantes todas las demás variables. Dado que el
valor p asociado es significativo (0.000807, menor que 0.05), este
coeficiente es estadísticamente significativo.

HGRAD: El coeficiente de HGRAD es 4.501e+02, lo que indica que por cada
incremento unitario en HGRAD, se espera que ROLL aumente en 4.501e+02
unidades, manteniendo constantes todas las demás variables. Dado que el
valor p asociado es significativo (1.52e-05, menor que 0.05), este
coeficiente es estadísticamente significativo.

INC: El coeficiente de INC es 4.275, lo que indica que por cada
incremento unitario en INC, se espera que ROLL aumente en 4.275
unidades, manteniendo constantes todas las demás variables. Dado que el
valor p asociado es significativo (5.59e-9, menor que 0.05), este
coeficiente es estadísticamente significativo.

El modelo tiene un R-cuadrado de 0.9621, lo que indica que el modelo
explica el 96.21\% de la variabilidad en la variable de respuesta ROLL.
El R-cuadrado ajustado, que toma en cuenta el número de predictores en
el modelo, es 0.9576.

El valor F es 211.5 con un valor p muy pequeño (\textless{} 2.2e-16), lo
que indica que el modelo en su conjunto es estadísticamente
significativo.

En general, este modelo parece ser un buen ajuste para los datos, ya que
tiene un R-cuadrado alto y todos los coeficientes son significativos.
Sin embargo, siempre es importante tener en cuenta las suposiciones de
la regresión lineal (por ejemplo, linealidad, homocedasticidad,
independencia de los errores) cuando se interpreta la salida del modelo.

\end{document}
